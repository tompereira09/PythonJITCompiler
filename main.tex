\documentclass[12pt]{article}

\title{Ficha Funções Quadráticas}
\author{Tomás Pereira}

\begin{document}
\maketitle

1.\\\\
R: A.\\



2.\\\\
h = $\frac{8\cdot x}{2}=96\equiv4x=96\equiv x=\frac{96}{4}\equiv x=24$\\\\
A(-4,  24) ; B(4, 24)\\\\
$f(x)=ax^2$\\
$24=a4^2\equiv24=16a\equiv16a=24\equiv a=\frac{24}{16}\equiv a=\frac{3}{2}$\\\\
R: B.\\

3.\\\\
$9=x^2$\\
x = $\pm\sqrt{9}$ ; $x\,>\,0$\\
C(3, 9)\\
A[OBC] = $3\cdot9=27$\\
A[AO] = $\frac{3\cdot9}{2}=\frac{27}{2}=13.5$\\
A[ACOB] = $27+13.5=40.5$\\\\
R: 40.5.\\\\\\

4.\\\\
A(3, 18)\\\\
$f(x)=2x^2$\\
$y=2\cdot3^2\equiv y =2\cdot9\equiv y=18$\\
$\overline{OA}=3$\\
$\overline{AB}=18$\\
A[OAB] = $\frac{3\cdot18}{2}=\frac{54}{2}=27$\\\\
R: C.\\
\\
5.1\\
R: 90 metros.\\\\
5.2\\
$d(t)=at^2\equiv40=a\cdot10^2\equiv40=100a\equiv a=\frac{40}{100}\equiv a=\frac{2}{5}$\\\\
R: C.\\

6.\\\\
$\overline{OA}\,=\,10$\\
$\overline{AB}\,=\,300$\\\\
$f(x)=3x^2$\\
$y=3\cdot10^2$\\
$y=3\cdot100$\\
$y=300$\\\\
A[OAB] = $\frac{10\cdot300}{2}=\frac{3000}{2}=1500$\\
A não sombreada = $1500-100=50'0$\\\\
R: A área da parte não sombreada mede 500 unidades.\\\\\\\\

7.\\\\
$\overline{OA}\,=\,4$\\
$\overline{OB}\,=\,\overline{AB}$\\\\
$f(x)=4x^2$\\
$y=4\cdot2^2$\\
$y=4\cdot4$\\
$y=16$\\\\
$h = 16$\\
A[OAB] = $\frac{4\cdot16}{2}=\frac{64}{2}=32$\\\\
R: A área do triângulo OAB mede 32 unidades.\\

8.\\\\
$\overline{CB}\,=\,2$\\\\
$f(x)=2x^2$\\
$y=2\cdot2^2$\\
$y=2\cdot4$\\
$y=8$\\\\
C(0, 8)\\\\
$\overline{CO}\,=\,8$\\\\
D(2, 0)\\\\
A[BCDO] = $2\cdot8=16$\\
A[ABD] = $\frac{2\cdot8}{2}=16$\\
A[ABCO] = $16+8=24$\\\\
R: A área do trapézio ABCO mede 24 unidades.\\\\\\


9.\\\\
$f(x)=x^2$\\
$g(x)=-x^2$\\\\
B(0, 2)\\\\
$f(\sqrt{3})=3$\\
$g(2)=-2^2=-4$\\\\
$3+(-4)=3-4=-1$\\\\
R: O número é -1.\\


10.\\\\
R: A.\\

11.\\\\
$f(x)=-2x^2$\\
$f(2)=-2\cdot2^2\equiv f(2)=-2\cdot4\equiv f(2)=-8$\\
P(2, -8)\\\\
R: Tendo em conta que a função g passa na origem do referencial, a resposta certa é a B.\\\\\\\\\\\\\\\\\\\\\\

12.1\\\\
R: A e B.\\

12.2\\\\
Yb = $2\cdot2^2=2\cdot4=8$\\
Yc = $\frac{4}{2}=2$\\
h = $8-2=6$\\
E(2, 2)\\
A[ABDE] = $2\cdot h=2\cdot6=12$\\
A[BEC] = $\frac{2\cdot h}{2}=\frac{2\cdot6}{2}=\frac{12}{2}=6$\\
A[ABCD] = $12+6=18$\\












\end{document}
